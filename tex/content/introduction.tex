\chapter{Introduction}

In this thesis we investigate an algorithm for finding approximate solutions for the \gls{btsp} and combine them with other results to obtain an approximation algorithm for the \gls{btspp}.

% Introduction

% Motivation
% Herleitung mit Beweisen
\section{Motivation}
My originally problem was related to the question if there is an algorithm for finding the shortest paths in a network (e.g.\ a railway network) which draws advantage of a particular index ordering of it's crossing points.
Let's say we have a network which can be represented as graph of edges and vertices. When storing the network with all its distances we need to give indices (integers starting from 0) to the vertices. Since the network is finite there is a finite constant \(c\) such that the distance of any two vertices can be bounded from above by \(c\) times the difference of their indices. The idea was to assign the indices in the way that this constant \(c\) is minimal.
This minimization problem turned out to be equivalent to the \gls{btspp}.