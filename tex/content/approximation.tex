\chapter{Approximation}

A 2-approximation algorithm for finding a hamiltonian cycle in the square of a biconnected graph was described by Parker and Rardin in~\cite{ParkerRardin1984}. \cite{alstrup2018hamiltonian} gives an improved version with a runtime linear in the number of edges in the graph.

\section{An approximation algorithm for BTSP}
In this section we compare the algorithms from~\cite{ParkerRardin1984} and~\cite{alstrup2018hamiltonian} and discuss the actual implementation.

On a very abstract level the algorithms contain of three steps:
\begin{enumerate}
  \item find a bottleneck optimal biconnected subgraph
  \item compute an open ear decomposition of that subgraph
  \item find an approximate solution using the open ear decomposition\label{enum:approximate}
\end{enumerate}

\subsection{Finding a bottleneck optimal biconnected subgraph}
The bottleneck optimal biconnected subgraph is the graph whose longest edge is shortest among all biconnected subgraphs.
Parker and Rardin~\cite{ParkerRardin1984} presented an easy implementable algorithm for finding such a subgraph: First all edges are sorted ascending in their weight and then edges are picked from the list until the graph is biconnected. Sorting costs \(\mathcal{O}(\abs{E} \log \abs{E})\) and the \(\mathcal{O}(\abs{E})\) checks for biconnectivity can be performed in \(\mathcal{O}(\abs{E}^2)\) when using Schmidts algorithm~\cite{schmidt2013}. This leads to a overall runtime of \(\mathcal{O}(\abs{E}^2)\).

In contrary in~\cite{alstrup2018hamiltonian} the much more sophisticated algorithm from~\cite{han1995} comes into play. This has two advantages over the naive approach:
\begin{enumerate}
  \item It has linear runtime.
  \item The algorithm produces a minimally biconnected subgraph.
\end{enumerate}
The second feature is a requirement for the algorithm presented to achieve step~\ref{enum:approximate}.

However, the algorithm by Han, Kelsen, Ramachandran and Tarjan comes with some drawbacks:
\begin{enumerate}
  \item The linear runtime comes with large constants.
  \item The algorithm is very bulky and craves very complex memory structures to obtain the proposed linear runtime in practical implementation.
\end{enumerate}

Therefore, the actual implementation sticks to the naive approach with some improvements to meet the requirement for the step~\ref{enum:approximate} algorithm.

\begin{algorithm}
  \caption{Finding a bottleneck optimal minimally biconnected subgraph}\label{alg:biconnected_subgraph}
  \begin{algorithmic}
 
  \end{algorithmic}
  \end{algorithm}

% Algorithmus einfügen

\subsection{Computing an open ear decomposition}
\subsection{Finding an approximate solution}

\section{An approximation algorithm for BTSPP}